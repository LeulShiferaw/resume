\documentclass{article}

\usepackage{amsmath}
\usepackage{amssymb}

\title{Problem 8.2}
\author{Leul Shiferaw}
\date{23-11-16}

\begin{document}
	\pagenumbering{gobble}
	\maketitle
	\newpage
	\pagenumbering{arabic}
	\large
	Q1.
	\begin{align*}
		f &= n^4\\
		g &= n^3log(n)\\
		n &>_a log(n)\\
		n^3*n &>_a n^3*log(n)\\
		n^4 &>_a n^3log(n)\\
		\therefore f &\in \Omega (g)		
	\end{align*}
	Q2.
	\begin{align*}		
		f &= n + n^2\\
		g &= n + 3n^3\\
		&\lim_{n\to\infty}(\frac{f}{g})\\
		&\lim_{n\to\infty}\frac{n^2+n}{3n^3+n}=0\\
		&\therefore f\in O(g)
	\end{align*}
	Q3.
	\begin{align*}
		f &= e^{n^2}\\
		g &= 2^n\\
		&\lim_{n\to\infty}\frac{e^{n^2}}{2^n}=\infty\\
	\end{align*}
	Since e is greater than 2 and $n^2 > n$ the limit goes to infinity
	\begin{align*}
		&\therefore f\in\Omega(g)
	\end{align*}
	Q4.
	\begin{align*}
		f &= n\\
		g &= sin(n)
	\end{align*}
	sin(n) is bounded between 0 and 1.
	\begin{align*}
		\therefore f\in\Omega(g)
	\end{align*}
	\newpage
	Q5.
	\begin{align*}
		&f = 12\\
		&g = 24\\
		&g = 2f\\
		&\therefore f\in\Theta(g)
	\end{align*}
	Q6.
	\begin{align*}
		&f = \frac{n^2}{log(n)}\\
		&g = n^4 + log(n)\\
		&\lim_{n\to\infty}{f/g}\\
		&\lim_{n\to\infty}{\frac{n^2}{n^4log(n)}}=0\\
		&\therefore f\in O(g)
	\end{align*}
	Q7.
	\begin{align*}
		&f = ln(n)\\
		&g = log(n)\\
		&f = \frac{log(n)}{log(e)}\\
		&f = \frac{g}{log(e)}\\
		&\therefore f\in\Theta(g)
	\end{align*}
	Q8.
	\begin{align*}
		&f = 5^n\\
		&g = n!
	\end{align*}
	$5^n$ is always multiplying 5, but n! starts from n and goes down. Therefore asymptotically n! will be greater than $5^n$
	\begin{align*}
		&\therefore f\in O(n!)
	\end{align*}
	\newpage
	Q9.
	\begin{align*}
		&f = 9^n\\
		&g = n^{log_n(8^n)}\\
		&g = 8^n\\
		&\lim_{n\to\infty}\frac{9^n}{8^n}\\
		&\lim_{n\to\infty}1.125^n=\infty\\
		&\therefore f\in\Omega(g)
	\end{align*}
	Q10.
	\begin{align*}
		&f = (n^3 + 3)^2\\
		&g = (3n^3 + 2)^2\\
		&f = n^6 + ...\\
		&g = 9n^6 + ...\\
		&\lim_{n\to\infty}\frac{f}{g} = 1/9\\
		&\therefore f\in\Theta(g)
	\end{align*}
\end{document}